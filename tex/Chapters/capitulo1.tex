\chapter{PROBABILIDAD}
\printchaptertableofcontents

En nuestras conversaciones cotidianas, el término \emph{probabilidad} es una medida de la creencia de que un evento futuro pueda ocurrir. Aceptamos esto como una interpretación significativa y práctica de probabilidad pero buscamos un concepto más claro de su contexto, de cómo se mide y cómo ayuda a hacer inferencias.

El concepto de probabilidad es necesario para trabajar con mecanismos físicos, biológicos o sociales que generan observaciones que no se pueden predecir con certeza. Por ejemplo, la presión sanguínea de una persona en un punto dado en el tiempo no se puede predecir con certeza y nunca sabemos la carga exacta que un puente resistirá antes de desplomarse en un río. Eventos de esa naturaleza no se pueden predecir con certeza, pero la frecuencia relativa con la que ocurren en una larga serie de intentos es a veces sorprendentemente estable. Los eventos que poseen esta propiedad reciben el nombre de eventos \emph{aleatorios} o \emph{estocásticos}.

Esta frecuencia relativa a largo plazo proporciona una medida intuitivamente significativa de nuestra creencia de que ocurrirá un evento aleatorio si se hace una observación futura. Es imposible, por ejemplo, predecir con certeza que una moneda normal caiga de cara en un solo tiro, pero estaríamos dispuestos a decir con un grado razonable de confianza que la fracción de caras en una larga serie de intentos sería muy cercana a $0.5$. Que esta frecuencia relativa se use comúnmente como medida de la creencia en el resultado de un solo tiro es evidente cuando consideramos la probabilidad desde el punto de vista de un jugador. Éste arriesga dinero en un solo tiro de una moneda, no en una larga serie de tiros. La frecuencia relativa de que aparezca una cara en una larga serie de tiros, que un jugador llama probabilidad de una cara, le da una medida de la probabilidad de ganar en un solo tiro. Si la moneda estuviera sesgada y diera $90\%$ de caras en una serie larga, el jugador consideraría que la probabilidad de cara es $0.9$ y confiaría en ese resultado en un solo intento.

El concepto de frecuencia relativa de probabilidad, aunque intuitivamente significativo, no da una definición rigurosa de probabilidad. Se han propuesto muchos otros conceptos de probabilidad, incluyendo el de probabilidad subjetiva que permite que la probabilidad de un evento varíe dependiendo de la persona que efectúe la evaluación. No obstante, para nuestros fines, aceptamos una interpretación basada en la frecuencia relativa como medida significativa de nuestra idea de que ocurrirá un evento.

\section{Notación y conceptos básicos de conjuntos}

Para nombrar conjuntos, generalmente se emplean letras mayúsculas $A$, $B$, $C$, $D$, $\dots$, y para nombrar a los elementos de un conjunto, si es el caso, se emplean usualmente letras minúsculas $a, b, c, d, \dots$. Para decir que un objeto $x$ es elemento de un conjunto $A$, escribiremos $x \in A$ y leemos: $x$ es elemento de $A$ o $x$ está en $A$ o $x$ pertenece a $A$. Para decir que un objeto $x$ no es elemento de un conjunto $A$, escribiremos $x \notin A$, y leemos: $x$ no es elemento de $A$ o $x$ no está en $A$ o $x$ no pertenece a $A$.

\subsection{Subconjuntos e igualdad de conjuntos}

\begin{definicion}{}{}
    Sean $A$ y $B$ conjuntos. Decimos que $A$ es subconjunto de $B$, y escribimos $A \subset B$, si cada elemento de $A$, es también elemento de $B$.
\end{definicion}

Observemos que la negación de que $A$ es subconjunto de $B$, es la negación de que cada elemento de $A$ es también elemento de $B$. Por tanto, $A$ no es subconjunto de $B$ si existe al menos un elemento de $A$ que no es elemento de $B$. En lugar de $A \subset B$ también se escribe $B \supset A$, y en cualquier caso se lee: $A$ es subconjunto de $B$ o $A$ está contenido en $B$ o $B$ contiene a $A$. Para decir que $A$ no es subconjunto de $B$ escribimos $A \not \subset B$ o $B \not \supset A$, y en cualquier caso se lee: $A$ no es subconjunto de $B$ o $A$ no está contenido en $B$ o $B$ no contiene a $A$. Simbólicamente se tiene que:
$$(A \subset B) \Longleftrightarrow (\forall x, x \in A \Longrightarrow x \in B)$$
y
$$(A \not \subset B) \Longleftrightarrow (\exists x \text{ tal que } x \in A \text{ y } x \notin B).$$
\begin{examplebox}{}{}
    Si
    \begin{align*}
        A & = \{x \mid x \text { es número real y }|x| \leq 3\}, \\
        B & = \{x \mid x \text { es número real y }|x| \leq 4\}
    \end{align*}
    entonces $A \subset B$ y $B \not \subset A$.
\end{examplebox}

\begin{examplebox}{}{}
    Si
    \begin{align*}
        A & = \{1, 2\}, \\
        B & = \{1, 2, 2, 1\}
    \end{align*}
    entonces $A \subset B$ y $B \subset A$.
\end{examplebox}

\begin{prop}{}{alalala}
    Si $A$, $B$ y $C$ son conjuntos, entonces:
    \begin{enumerate}[label=\roman*., topsep=6pt, itemsep=0pt]
        \item $A \subset A$.
        \item $(A \subset B$ y $B \subset C) \Longrightarrow A \subset C$.
    \end{enumerate}
    \tcblower
    \demostracion
    \begin{enumerate}[label=\roman*., topsep=6pt, itemsep=0pt]
        \item Es inmediato.
        \item Si $A \subset B$ y $B \subset C$, entonces $(x \in A \Longrightarrow x \in B)$ y $(x \in B \Longrightarrow x \in C)$, por tanto $(x \in A \Longrightarrow x \in C)$, de donde se sigue que $A \subset C$.
    \end{enumerate}
\end{prop}

\newpage

\begin{definicion}{}{}
    Sean $A$ y $B$ conjuntos. Decimos que $A$ es igual a $B$, y escribimos $A = B$, si $A \subset B$ y $B \subset A$.
\end{definicion}

Simbólicamente se tiene que:
$$(A = B) \Longleftrightarrow (A \subset B \text{ y } B \subset A).$$
Observemos que la negación de que los conjuntos $A$ y $B$ son iguales, lo que se expresa escribiendo $A \neq B$, y que se lee: $A$ no es igual a $B$ o $A$ es distinto de $B$, es la negación de que $[A \subset B$ y $B \subset A]$. En consecuencia,
$$[A \neq B] \Longleftrightarrow[A \not \subset B \text{ o } B \not \subset A].$$
Es decir,
$$[A \neq B] \Longleftrightarrow[(\exists x \text{ tal que } x \in A \text { y } x \notin B) \text { o }(\exists x \text{ tal que } x \in B \text{ y } x \notin A)].$$
\begin{examplebox}{}{}
    Si
    $$A = \left\{ x \mid x \text{ es número real y } x^2 - 1 = 0 \right\}$$
    y
    $$B = \{ -1,  1 \},$$
    entonces $A = B$.
\end{examplebox}

\begin{examplebox}{}{}
    Si
    $$A = \left\{ x \mid x \text{ es número primo y } x>2 \right\}$$
    y
    $$B = \left\{ x \mid x \text{ es número impar} \right\},$$
    entonces $A \subset B$ y $B \not\subset A$, por tanto $A \neq B$.
\end{examplebox}

\begin{prop}{}{}
    Si $A$, $B$ y $C$ son conjuntos, entonces:
    \begin{enumerate}[label=\roman*., topsep=6pt, itemsep=0pt]
        \item $A = A$.
        \item $A = B \Longrightarrow B = A$.
        \item $(A = B$ y $B = C) \Longrightarrow A = C$.
    \end{enumerate}
    \tcblower
    \demostracion
    \begin{enumerate}[label=\roman*., topsep=6pt, itemsep=0pt]
        \item $A \Longrightarrow A \subset A$ y $A \subset A \Longrightarrow A=A$.
        \item $A=B \Longrightarrow A \subset B$ y $B \subset A \Longrightarrow B \subset A$ y $A \subset B \Longrightarrow$ $B=A$.
        \item $A=B$ y $B=C \Longrightarrow(A \subset B$ y $B \subset A)$ y $(B \subset C$ y $C \subset B)$. Entonces $(A \subset B$ y $B \subset C)$ y $(C \subset B$ y $B \subset A)$. Por (ii) de la proposición \ref{prop:alalala}, se sigue $A \subset C$ y $C \subset A \Longrightarrow A=C$.
    \end{enumerate}
\end{prop}

\subsection{El conjunto vacío}

\begin{definicion}{}{}
    Para cada conjunto $A$, definimos el conjunto
    $$\varnothing_A = \{x \mid x \in A \text { y } x \neq x\},$$
    el cual es subconjunto de $A$ y se le llama subconjunto vacío (o nulo) de $A$.
\end{definicion}

\begin{prop}{}{}
    Si $A$ y $B$ son conjuntos, entonces
    $$\varnothing_A = \varnothing_B.$$

    \tcblower
    \demostracion Se deja como ejercicio al lector.
\end{prop}

\newpage

\begin{definicion}{}{}
    Al único conjunto que no posee elementos, se le llama conjunto vacío y se le denota por $\varnothing$.
\end{definicion}

El conjunto $\varnothing$ es subconjunto de cualquier conjunto, es decir, para cualquier conjunto $A$, $\varnothing \subset A$. Así que, si $A$ es conjunto no vacío, entonces al menos tiene como subconjuntos a $\varnothing$ y a $A$ mismo, los cuales se llaman \emph{subconjuntos impropios} de $A$.

\begin{definicion}{}{}
    Decimos que un conjunto $B$ es subconjunto propio de un conjunto $A$, si $B \neq \varnothing$, $B \subset A$ y $B \neq A$.
\end{definicion}

Observemos que
$$(B \subset A \text{ y } B \neq A) \Longleftrightarrow(B \subset A \text{ y } A \not \subset B).$$

\subsection{La unión e intersección de conjuntos}

\begin{definicion}{}{}
    Sean $A$ y $B$ conjuntos. Definimos la unión de $A$ y $B$, denotada por $A \cup B$, como el conjunto
    $$A \cup B = \{x \mid x \in A \text{ o } x \in B\}.$$
    La expresión $A \cup B$, se lee: $A$ unión $B$ o la unión de $A$ y $B$.
\end{definicion}

Simbólicamente se tiene que:
$$(x \in A \cup B) \Longleftrightarrow(x \in A \text{ o } x \in B)$$
y
$$(x \notin A \cup B) \Longleftrightarrow(x \notin A \text{ y } x \notin B).$$
\begin{examplebox}{}{}
    Si
    $$A = \{1,  3\}$$
    y
    $$B = \{0,  1,  2\},$$
    entonces $A \cup B = \{0,  1,  2,  3\}$.
\end{examplebox}

\begin{examplebox}{}{}
    Si $A = \varnothing$ y $B = \varnothing$, entendemos que
    $$A \cup B = \varnothing.$$
\end{examplebox}

\begin{prop}{}{}
    Si $A$, $B$ y $C$ son conjuntos, entonces:
    \begin{enumerate}[label=\roman*., topsep=6pt, itemsep=0pt]
        \item $A \cup A = A$.
        \item $A \cup B = B \cup A$.
        \item $\varnothing \cup A = A$.
        \item $(A \cup B) \cup C = A \cup(B \cup C)$.
        \item $A \subset A \cup B$ y $B \subset A \cup B$.
        \item $(A \subset C$ y $B \subset C) \Longleftrightarrow A \cup B \subset C$.
        \item $A \subset B \Longleftrightarrow A \cup B=B$.
        \item $A \subset B \Longrightarrow A \cup C \subset B \cup C$.
        \item $A \subset B$ y $C \subset D \Longrightarrow A \cup C \subset B \cup D$.
    \end{enumerate}
    \tcblower
    \demostracion Solo se demostrará (viii), los demás se dejan como ejercicio al lector.
    \begin{enumerate}[label=\roman*., topsep=6pt, itemsep=0pt]
        \item[viii.] $x \in A \cup C \Longrightarrow x \in A$ o $x \in C \Longrightarrow x \in B$ o $x \in C \Longrightarrow x \in B \cup C$.
    \end{enumerate}
\end{prop}

\newpage

\begin{definicion}{}{}
    Sean $A$ y $B$ conjuntos. Definimos la intersección de $A$ y $B$, denotada por $A \cap B$, como el conjunto
    $$A \cap B = \{x \mid x \in A \text { y } x \in B\}.$$
    La expresión $A \cap B$, se lee: $A$ intersección $B$ o la intersección de $A$ y $B$.
\end{definicion}

Simbólicamente se tiene que:
$$(x \in A \cap B) \Longleftrightarrow(x \in A \text { y } x \in B)$$
y
$$(x \notin A \cap B) \Longleftrightarrow(x \notin A \text { o } x \notin B).$$

\begin{examplebox}{}{}
    Si
    $$A = \{1,  2,  3\}$$
    y
    $$B = \{0,  1,  2\},$$
    entonces $A \cap B = \{1,  2\}$.
\end{examplebox}

\begin{examplebox}{}{}
    Si
    $$A = \{x \mid x \text{ es número par}\}$$
    y
    $$B = \{x \mid x \text{ es número impar}\},$$
    entonces $A \cap B = \varnothing$.
\end{examplebox}

\begin{prop}{}{}
    Si $A$, $B$ y $C$ son conjuntos, entonces:
    \begin{enumerate}[label=\roman*., topsep=6pt, itemsep=0pt]
        \item $A \cap A=A$.
        \item $A \cap B=B \cap A$.
        \item $\varnothing \cap A=\varnothing$.
        \item $(A \cap B) \cap C=A \cap(B \cap C)$.
        \item $A \cap B \subset A$ y $A \cap B \subset B$.
        \item $(C \subset A$ y $C \subset B) \Longleftrightarrow C \subset A \cap B$.
        \item $(A \subset B) \Longleftrightarrow(A \cap B=A)$.
        \item $A \subset B \Longrightarrow A \cap C \subset A \cap C$.
        \item $A \subset B$ y $C \subset D \Longrightarrow A \cap C \subset B \cap D$.
    \end{enumerate}
    \tcblower
    \demostracion Se proponen como ejercicio al lector.
\end{prop}

\begin{definicion}{}{}
    Sean $A$ y $B$ conjuntos. Decimos que $A$ y $B$ son ajenos, si $A \cap B = \varnothing$.
\end{definicion}

Observemos que el conjunto vacío es ajeno con cualquier conjunto.

La siguiente proposición enuncia las propiedades distributivas de la unión respecto de la intersección, y viceversa.

\begin{prop}{}{}
    Si $A$, $B$ y $C$ son conjuntos, entonces:
    \begin{enumerate}[label=\roman*., topsep=6pt, itemsep=0pt]
        \item $A \cap(B \cup C) = (A \cap B) \cup(A \cap C)$.
        \item $A \cup(B \cap C) = (A \cup B) \cap(A \cup C)$.
    \end{enumerate}
    \tcblower
    \demostracion
    \begin{enumerate}[label=\roman*., topsep=6pt, itemsep=0pt]
        \item Por definición de igualdad, debemos probar que
        $$A \cap(B \cup C) \subset(A \cap B) \cup(A \cap C)$$
        y que
        $$(A \cap B) \cup(A \cap C) \subset A \cap(B \cup C).$$
        En este caso podemos probar simultáneamente ambas contenciones. De esta forma, tenemos que
        \begin{align*}
            x \in A \cap(B \cup C) & \Longleftrightarrow x \in A \text { y } x \in(B \cup C) \\ 
            & \Longleftrightarrow x \in A \text { y }(x \in B \text { o } x \in C) \\ 
            & \Longleftrightarrow (x \in A \text { y } x \in B) \text { o }(x \in A \text { y } x \in C) \\ 
            & \Longleftrightarrow x \in A \cap B \text { o } x \in A \cap C \\ 
            & \Longleftrightarrow x \in(A \cap B) \cup(A \cap C)
        \end{align*}
        \item Para demostrar la igualdad, debemos verificar ambas contenciones. Primero, debemos probar que
        $$A \cup(B \cap C) \subset (A \cup B) \cap (A \cup C).$$
        Luego, debemos mostrar que
        $$(A \cup B) \cap (A \cup C) \subset A \cup(B \cap C).$$
        Pero también podemos proceder de la siguiente manera:
        \begin{align*}
            (A \cup B) \cap(A \cup C) & =[(A \cup B) \cap A] \cup[(A \cup B) \cap C] \\
            & =[A] \cup[(A \cap C) \cup(B \cap C)] \\
            & =[A \cup(A \cap C)] \cup(B \cap C) \\
            & =A \cup(B \cap C)
        \end{align*}
    \end{enumerate}
\end{prop}

\subsection{Diferencia de conjuntos}

\begin{definicion}{}{}
    Sean $A$ y $B$ conjuntos. Definimos la diferencia de $A$ y $B$, denotada por $A - B$, como el conjunto
    $$A - B = \{x \mid x \in A \text { y } x \notin B\}.$$
    La expresión $A - B$ se lee: $A$ menos $B$ o la diferencia de $A$ y $B$.
\end{definicion}

Simbólicamente se tiene que:
$$(x \in A - B) \Longleftrightarrow(x \in A \text{ y } x \notin B)$$
y
$$(x \notin A - B) \Longleftrightarrow(x \notin A \text{ o } x \in B).$$

\begin{examplebox}{}{}
    Si
    $$A = \{1,  2,  3,  4,  5,  6\} \quad \text{ y } \quad B = \{2,  4,  6,  8,  10\},$$
    entonces
    $$A - B = \{1,  3,  5\} \quad \text{ y } \quad B - A = \{8,  10\}.$$
\end{examplebox}

\begin{examplebox}{}{}
    Si
    $$A = \{x \mid x \text { es número natural}\} \quad \text{ y } \quad B = \{x \mid x \text { es número par}\},$$
    entonces 
    $$A - B = \{x \mid x \text { es número impar}\} \quad \text{ y } \quad B - A = \varnothing.$$
\end{examplebox}

De los ejemplos anteriores se deduce que, en general,
$$A - B \neq B - A;$$\newpage\noindent
es decir, que la diferencia de conjuntos no es conmutativa. Esto significa que el orden en el que se restan los conjuntos importa, ya que restar $B$ de $A$ no produce el mismo resultado que restar $A$ de $B$. Además, la diferencia de conjuntos no es asociativa, es decir, en general,
$$(A - B) - C \neq A - (B - C).$$
Esto implica que la forma en que agrupamos los conjuntos al restarlos también afecta el resultado.

\begin{prop}{}{}
    Si $A$ y $B$ son conjuntos, entonces:
    \begin{enumerate}[label=\roman*., topsep=6pt, itemsep=0pt]
        \item $A - A = \varnothing$.
        \item $A - \varnothing = A$.
        \item $A - B \subset A$.
        \item $A \subset B \Longleftrightarrow A - B = \varnothing$.
    \end{enumerate}
    \tcblower
    \demostracion Se dejan como ejercicio al lector.
\end{prop}

\subsection{Complemento de un conjunto}

\begin{definicion}{}{}
    Sean $A$ y $B$ conjuntos. Definimos el complemento de $A$ relativo a $B$, denotado por $\mathcalm{C}_B(A)$, como el conjunto
    $$\mathcalm{C}_B(A) = B - A.$$
\end{definicion}

Siempre que trabajamos con conjuntos, podemos suponer que estos son subconjuntos de otro conjunto ``más grande'' denominado \emph{conjunto universal}. Tal conjunto universal depende del discurso, esto es, de los conjuntos con que se trabaje en una situación dada. En general llamaremos $S$ al conjunto universal de un discurso, y esto significa que los conjuntos que se mencionen en el discurso, serán subconjuntos de $S$. Lo anterior permitirá simplificar la escritura cuando hablemos del complemento de un conjunto $A$, relativo al conjunto universal $S$; pues en lugar de $\mathcalm{C}_S(A)$, escribiremos simplemente $\mathcalm{C}(A)$ o $A^C$. Por tanto
$$A^C = S - A.$$
Vamos a convenir en escribir
$$x \in A^C \Longleftrightarrow x \notin A,$$
en lugar de
$$x \in A^C \Longleftrightarrow x \in S \text{ y } x \notin A,$$
y por tanto
$$x \notin A^C \Longleftrightarrow x \in A.$$

\begin{prop}{}{}
    Si $A$ es un conjunto y $S$ es el conjunto universal, entonces:
    \begin{enumerate}[label=\roman*., topsep=6pt, itemsep=0pt]
        \item $A \cup A^C = S$.
        \item $A \cap A^C = \varnothing$.
        \item $A - A^C = A$.
        \item $\left(A^C\right)^C = A$.
        \item $\varnothing^C = S$.
        \item $S^C = \varnothing$.
        \item $A \cup S = S$.
        \item $A \cap S = A$.
        \item $A - S = \varnothing$.
        \item $A - B = A \cap B^C$.
        \item $A \subset B \Longleftrightarrow B^C \subset A^C$.
        \item $A \subset B^C \Longleftrightarrow B \subset A^C$.
    \end{enumerate}
    \tcblower
    \demostracion Solo se demostrará (i), los demás incisos se dejan como ejercicio al lector.
    \begin{enumerate}[label=\roman*., topsep=6pt, itemsep=0pt]
        \item Por definición,
        $$A^C = \{x \in S \mid x \notin A\}.$$
        Ahora, consideremos
        $$A \cup A^C = \left\{x \in S \mid x \in A \text{ o } x \in A^C\right\}.$$
        Dado que $A$ y $A^C$ cubren todos los elementos del conjunto universal $S$, cualquier elemento $x \in S$ estará en $A$ o en $A^C$. Por lo tanto
        $$A \cup A^C = S.$$
    \end{enumerate}
\end{prop}

El siguiente teorema es conocido como el Teorema de De Morgan. Este teorema establece las relaciones entre las operaciones de unión e intersección de conjuntos y sus complementos, proporcionando una herramienta fundamental en la teoría de conjuntos y la lógica matemática.

\begin{theorem}{}{}
    Si $A$ y $B$ son conjuntos y $S$ es el conjunto universal, entonces:
    \begin{enumerate}[label=\roman*., topsep=6pt, itemsep=0pt]
        \item $(A \cup B)^C = A^C \cap B^C$.
        \item $(A \cap B)^C = A^C \cup B^C$.
    \end{enumerate}
    \tcblower
    \demostracion
    \begin{enumerate}[label=\roman*., topsep=6pt, itemsep=0pt]
        \item Para demostrar que $(A \cup B)^C = A^C \cap B^C$, consideremos un elemento $x$ y analicemos su pertenencia a los conjuntos involucrados:
        \begin{align*}
            x \in(A \cup B)^C & \Longleftrightarrow x \notin A \cup B \\ 
            & \Longleftrightarrow x \notin A \text{ y } x \notin B \\ 
            & \Longleftrightarrow  x \in A^C \text { y } x \in B^C \\ 
            & \Longleftrightarrow x \in A^C \cap B^C
        \end{align*}
        \item Para demostrar que $(A \cap B)^C = A^C \cup B^C$, consideremos un elemento $x$ y analicemos su pertenencia a los conjuntos involucrados:
        \begin{align*}
            x \in(A \cap B)^C & \Longleftrightarrow x \notin A \cap B \\ 
            & \Longleftrightarrow x \notin A \text{ y } x \notin B \\ 
            & \Longleftrightarrow  x \in A^C \text { o } x \in B^C \\ 
            & \Longleftrightarrow x \in A^C \cup B^C
        \end{align*}
    \end{enumerate}
\end{theorem}

\begin{prop}{}{}
    Si $A$, $B$, $C$ y $D$ son conjuntos y $S$ es el conjunto universal, entonces:
    \begin{enumerate}[label=\roman*., topsep=6pt, itemsep=0pt]
        \item $A - B = A - (A \cap B)$.
        \item $(A \cap B = \varnothing$ y $A \cup B = S) \Longrightarrow B = A^C$.
        \item $A - B = A \Longleftrightarrow B - A = B$.
        \item $(A \cup B) - (A \cap B) = (A - B) \cup(B - A)$.
    \end{enumerate}
    \tcblower
    \demostracion Se dejan como ejercicio al lector.
\end{prop}

\newpage

\subsection{Diagramas de Venn-Euler}

En los diagramas de Venn-Euler, a un conjunto no vacío se le representa con una figura cerrada, dentro de un rectángulo que representa el conjunto universal y se utilizan para comparar, contrastar y analizar la intersección, la unión y la diferencia de dichos conjuntos.
\begin{figure}[h!]
    \centering
    \begin{tikzpicture}
        \coordinate (A) at (-2.2,0);
        \coordinate (B) at (3,0.5);
        \coordinate (C) at (0.5,-0.3);
        \draw[thick] (-4.5,-2) rectangle (4.5,2);
        \draw[cw0] (A) circle (1.2cm) node {$A$};
        \draw[cw0!75] (B) circle (0.9cm) node {$B$};
        \draw[cw0!50] (C) circle (0.6cm) node {$C$};
    \end{tikzpicture}
    \caption{Ejemplo de diagrama de Venn-Euler que muestra tres conjuntos $A$, $B$ y $C$ dentro de un conjunto universal $S$. El conjunto $A$ está representado por el círculo más grande a la izquierda, el conjunto $B$ por el círculo mediano a la derecha, y el conjunto $C$ por el círculo más pequeño en el centro.}
\end{figure}

\noindent Las intersecciones entre las figuras se utilizan para mostrar la relación entre estos conjuntos, ya sea para señalar la presencia de elementos comunes o para evidenciar su exclusividad. Además, podemos hacer diagramas de las operaciones antes vistas:
\def\firstcircle{(0,0) circle (1.5cm)}
\def\secondcircle{(0:2cm) circle (1.5cm)}

\colorlet{circle edge}{cw0}
\colorlet{circle area}{cw0!70}

\tikzset{filled/.style={fill=circle area, draw=circle edge, thick},outline/.style={draw=circle edge, thick}}

\begin{figure*}[h!]
    \centering
    \subfloat[$A \cup B$]{
    \begin{tikzpicture}
        \draw[filled] \firstcircle node[white] {$A$}
        \secondcircle node[white] {$B$};
    \end{tikzpicture}
    } \hfill
    \subfloat[$A \cap B$]{
    \begin{tikzpicture}
        \begin{scope}
            \clip \firstcircle;
            \fill[filled] \secondcircle;
        \end{scope}
        \draw[outline] \firstcircle node {$A$};
        \draw[outline] \secondcircle node {$B$};
    \end{tikzpicture}
    } \hfill
    \subfloat[$A-B$]{
    \begin{tikzpicture}
        \begin{scope}
            \clip \firstcircle;
            \draw[filled, even odd rule] \firstcircle node[white] {$A$}
            \secondcircle;
        \end{scope}
        \draw[outline] \firstcircle
        \secondcircle node {$B$};
    \end{tikzpicture}
    } \\
    \subfloat[$B-A$]{
    \begin{tikzpicture}
        \begin{scope}
            \clip \secondcircle;
            \draw[filled, even odd rule] \firstcircle
            \secondcircle node[white] {$B$};
        \end{scope}
        \draw[outline] \firstcircle node {$A$}
        \secondcircle;
        \node at (0,-2) {~};
    \end{tikzpicture}
    } \hfill
    \subfloat[$A^C$]{
    \begin{tikzpicture}
        \begin{scope}
            \fill[cw0!70] (-2.5,-2) rectangle (2.5,2);
            \draw[thick,cw0] (-2.5,-2) rectangle (2.5,2);
            \fill[white] (0,0) circle (1.5cm);
            \node at (0,0) {$A$};
            \draw[cw0!70,thick] (0,0) circle (1.5cm);
        \end{scope}
    \end{tikzpicture}
    } \hfill
    \subfloat[$A \subset B$]{
    \begin{tikzpicture}
        \begin{scope}
            \draw[thick,cw0] (-2.5,-2) rectangle (2.5,2);
            \node at (0.8,0.8) {$B$};
            \draw[cw0,thick] (0,0) circle (1.5cm);
            \node at (-0.5,-0.5) {$A$};
            \draw[cw0,thick] (-0.5,-0.5) circle (0.6cm);
        \end{scope}
    \end{tikzpicture}
    }
    \caption{Diagramas de Venn que ilustran varias operaciones de conjuntos. (a) La unión de $A$ y $B$, representada por el área sombreada que cubre ambos conjuntos. (b) La intersección de $A$ y $B$, mostrada como el área común sombreada entre ambos conjuntos. (c) La diferencia $A - B$, que es el área de $A$ excluyendo la intersección con $B$. (d) La diferencia $B - A$, que es el área de $B$ excluyendo la intersección con $A$. (e) El complemento de $A$, $A^C$, que es el área fuera del conjunto $A$ dentro del conjunto universal. (f) La relación de subconjunto $A \subset B$, donde $A$ está completamente contenido dentro de $B$.}
\end{figure*}

\subsection{Conjunto potencia y la cardinalidad de un conjunto}

\begin{definicion}{}{}
    Sea $A$ un conjunto. Se define el conjunto potencia de $A$, denotado por $\wp(A)$ o por $2^A$, como el conjunto
    $$\wp(A) = \{B \mid B \subset A\}.$$
\end{definicion}

\newpage

Observemos que
\begin{enumerate}
    \item Por definición, $B \subset A \Longleftrightarrow B \in \wp(A)$.
    \item Para cualquier conjunto $A$, se cumple que $\varnothing \in \wp(A)$ y $A \in \wp(A)$.
    \item $a \in A \Longleftrightarrow \{a\} \in \wp(A)$.
\end{enumerate}

\begin{examplebox}{}{}
    Si $A = \{0, 1\}$, entonces
    $$\wp(A) = \big\{\varnothing, \{0\}, \{1\}, A \big\}.$$
\end{examplebox}

\begin{theorem}{}{}
    Si $X$ y $Y$ son conjuntos, entonces
    $$X \subset Y \Longleftrightarrow \wp(X) \subset \wp(Y).$$
    \tcblower
    \demostracion Procedamos por casos.
    \begin{enumerate}[label=\roman*., topsep=6pt, itemsep=0pt]
        \item Probemos primero que $X \subset Y \Longrightarrow \wp(X) \subset$ $\wp(Y)$: Si $A \in \wp(X)$, entonces $A \subset X$, y como por hipótesis $X \subset Y$, entonces $A \subset Y$, entonces $A \in \wp(Y)$.
        \item Ahora probaremos que $\wp(X) \subset \wp(Y) \Longrightarrow X \subset Y$: Si $x \in X$, entonces $\{x\} \in \wp(X)$, y como por hipótesis $\wp(X) \subset \wp(Y)$, entonces $\{x\} \in \wp(Y)$, entonces $x \in Y$.
    \end{enumerate}
\end{theorem}

La \emph{cardinalidad} de un conjunto permite cuantificar y comparar el tamaño de diferentes colecciones de elementos, tanto finitos como infinitos.

\begin{definicion}{}{}
    Dado un conjunto $A$, definimos su \emph{cardinalidad}, denotado por $|A|$, de la siguiente manera:
    \begin{enumerate}[label=\roman*., topsep=6pt, itemsep=0pt]
        \item Si $A$ es \emph{finito}, entonces $|A|$ es simplemente el número de elementos en $A$.
        \item Si $A$ es \emph{infinito}, se define su cardinalidad mediante el concepto de \emph{equipotencia} o correspondencia biyectiva con otros conjuntos.
    \end{enumerate}
\end{definicion}
\infoBulle{Para un estudio detallado, recomendamos las obras clásicas \emph{Naïve Set Theory} de Paul R. Halmos y \emph{Set Theory and Its Philosophy} de Michael Potter.}

Mientras para conjuntos finitos la noción es intuitiva y aritmética, para conjuntos infinitos requiere herramientas más sofisticadas de la teoría de conjuntos. Aunque en esta obra no profundizaremos en los detalles de la \emph{equipotencia} ni en la jerarquía de \emph{cardinales transfinitos} (como $\aleph_1$, $\aleph_2$, etc.), es importante señalar que estos conceptos revolucionaron la comprensión matemática del infinito.

\begin{examplebox}{}{}
    \begin{enumerate}[topsep=6pt, itemsep=0pt]
        \item El conjunto vacío, carente de elementos, satisface $| \varnothing | = 0$ por definición.
        \item Si $A = \{a, b, c, d, e\}$, entonces $|A| = 5$.
        \item Si $C = \NN$, su cardinalidad es infinita y se denota como $\aleph_0$ (aleph-cero).
    \end{enumerate}
\end{examplebox}

\begin{definicion}{}{}
    Un conjunto $A$ es \emph{numerable} si $|A| = |\NN|$.
\end{definicion}

La numerabilidad implica que, aunque un conjunto sea infinito, sus elementos pueden ser sistemáticamente enumerados. Este resultado, establecido por Georg Cantor, desafía la intuición al mostrar que conjuntos aparentemente “más grandes” como los enteros o racionales comparten la misma cardinalidad que los naturales.

\begin{definicion}{}{}
    Si el conjunto $A$ es finito y numerable, entonces $A$ es un \emph{conjunto discreto}.
\end{definicion}

\section{Espacio muestral}

Al inicio de este capítulo nos referimos al \emph{experimento} de lanzar una moneda y observar qué cara aparecía. Usaremos este término para referirnos tanto a situaciones incontrolables (como el precio diario de una acción) como a observaciones en condiciones controladas de laboratorio. Tenemos la siguiente definición:

\newpage

\begin{definicion}{}{}
    Un \emph{experimento} es el proceso por medio del cual se hace una observación.
\end{definicion}

Ejemplos de experimentos incluyen el tiro de monedas y dados, medir el cociente de inteligencia (IQ) de una persona o determinar el número de bacterias por centímetro cúbico en una porción de alimento procesado.

\begin{definicion}{}{}
    Al conjunto de todos los resultados posibles de un experimento se le llama \emph{espacio muestral} y se representa con el símbolo $S$.
\end{definicion}

A cada resultado en un espacio muestral se le llama \emph{elemento} o \emph{miembro} del espacio muestral. Si el espacio muestral tiene un número finito de elementos, podemos listar los miembros separados por comas y encerrarlos entre llaves. Por consiguiente, el espacio muestral $S$, de los resultados posibles cuando se lanza una moneda al aire, se puede escribir como
$$S = \left\{ \text{sol}, \text{águila} \right\} .$$
\begin{examplebox}{}{primer_ejemplo}
    Considere el experimento de lanzar un dado. Si nos interesara el número que aparece en la cara superior, el espacio muestral sería
    $$S_1 = \left\{ 1, 2, 3, 4, 5, 6 \right\}.$$
    Si sólo estuviéramos interesados en si el número es par o impar, el espacio muestral sería simplemente
    $$S_2 = \left\{ \text{par}, \text{impar} \right\}.$$
\end{examplebox}

El ejemplo anterior ilustra el hecho de que se puede usar más de un espacio muestral para describir los resultados de un experimento. En este caso, $S_1$ brinda más información que $S_2$. Si sabemos cuál elemento ocurre en $S_1$, podremos indicar cuál resultado tiene lugar en $S_2$; no obstante, saber lo que pasa en $S_2$ no ayuda mucho a determinar qué elemento ocurre en $S_1$. En general, lo deseable sería utilizar un espacio muestral que proporcione la mayor información acerca de los resultados del experimento. En algunos experimentos es útil listar los elementos del espacio muestral de forma sistemática utilizando un \emph{diagrama de árbol}.

\begin{examplebox}{}{}
    Un experimento consiste en lanzar una moneda y después lanzarla una segunda vez si sale sol. Si en el primer lanzamiento sale águila, entonces se lanza un dado una vez. Para listar los elementos del espacio muestral que proporciona la mayor información construimos el diagrama de árbol de la figura \ref{fig:diagrama_de_arbol}.
    \begin{center}
        \begin{tikzpicture}[grow=right,-latex,>=angle 60]
            \node {\makecell{Lanzar una \\ moneda}}
                child {node {\makecell{águila \\ (A)}}
                    child {node {6}
                        child[draw=white,-] {node{A6}}
                    }
                    child {node {5}
                        child[draw=white,-] {node{A5}}
                    }
                    child {node {4}
                        child[draw=white,-] {node{A4}}
                    }
                    child {node {3}
                        child[draw=white,-] {node{A3}}
                    }
                    child {node {2}
                        child[draw=white,-] {node{A2}}
                    }
                    child {node{1}
                        child[draw=white,-] {node{A1}}
                    }
                }
                child {node {\makecell{sol \\ (S)}}
                    child {node{A}
                        child[draw=white,-] {node{SA}}
                    }
                    child {node{S}
                        child[draw=white,-] {node{SS}}
                    }
                };
            \end{tikzpicture}
        \captionsetup*[figure]{hypcap=false}
        \captionof{figure}{Diagrama de árbol}\label{fig:diagrama_de_arbol}
    \end{center}
    Si empezamos con la rama superior izquierda y nos movemos a la derecha a lo largo de la primera trayectoria, obtenemos el elemento SS, que indica la posibilidad de que ocurran caras en dos lanzamientos sucesivos de la moneda. De igual manera, el elemento A3 indica la posibilidad de que la moneda muestre un águila seguida por un 3 en el lanzamiento del dado. Al seguir todas las trayectorias, vemos que el espacio muestral es
    $$S = \left\{ \text{SS, SA, A1, A2, A3, A4, A5, A6} \right\} .$$
\end{examplebox}

\newpage

\begin{examplebox}{}{}
    Consideremos un lote de tornillos del cual se extrae uno sin reemplazo para verificar su diámetro. Si el diámetro está dentro del rango especificado, se considera un \emph{éxito}, representado por $E$; en caso contrario, se obtiene un \emph{fracaso}, denotado por $F$. El proceso continúa hasta que ocurre el primer fracaso. En este caso, el espacio muestral está dado por:
    $$S = \{ F, EF, EEF, EEEF, \dots \}.$$
\end{examplebox}

Muchos de los conceptos de este capítulo se ilustran mejor con ejemplos que involucran el uso de dados y cartas. Es particularmente importante utilizar estas aplicaciones al comenzar el proceso de aprendizaje, ya que facilitan el flujo de esos conceptos nuevos en ejemplos científicos y de ingeniería.

\begin{definicion}{}{}
    Un \emph{espacio muestral discreto} es aquel que está formado ya sea por un número finito o uno contable de puntos muestrales distintos.
\end{definicion}

\section{Eventos}

En cualquier experimento dado, podríamos estar interesados en la ocurrencia de ciertos \emph{eventos}, más que en la ocurrencia de un elemento específico en el espacio muestral. Por ejemplo, quizás estemos interesados en el evento $A$, en el cual el resultado de lanzar un dado es divisible entre 3. Esto ocurrirá si el resultado es un elemento del subconjunto $A = \{3, 6\}$ del espacio muestral $S_1$ del ejemplo \ref{examplebox:primer_ejemplo}.

Para cada evento asignamos un conjunto de elementos, que constituye un
subconjunto del espacio muestral. Este subconjunto representa la totalidad de los elementos para los que el evento es cierto.

\begin{definicion}{}{}
    Un \emph{evento} es un subconjunto de un espacio muestral. Además, en particular, un \emph{evento simple} no se puede descomponer. Cada evento simple corresponde a un y sólo un elemento del espacio muestral.
\end{definicion}

\begin{examplebox}{}{}
    Dado el espacio muestral $S = \{t \mid t \geq 0\}$, donde $t$ es la vida en años de cierto componente electrónico, el evento $A$ de que el componente falle antes de que finalice el quinto año es el subconjunto $A = \{t \mid 0 \leq t < 5\}$.
\end{examplebox}

Es posible concebir que un evento puede ser un subconjunto que incluye todo el espacio muestral $S$, o un subconjunto de $S$ que se denomina conjunto vacío y se denota con el símbolo $\varnothing$, que no contiene ningún elemento. Por ejemplo, si en un experimento biológico permitimos que $A$ sea el evento de detectar un organismo microscópico a simple vista, entonces $A = \varnothing$. También, si
$$B = \left\{ x \mid x \text{ es un factor par de } 7 \right\},$$
entonces $B$ debe ser el conjunto vacío, pues los únicos factores posibles de 7 son los números nones 1 y 7.

Considere un experimento en el que se registran los hábitos de tabaquismo de los empleados de una empresa industrial. Un posible espacio muestral podría clasificar a un individuo como no fumador, fumador ocasional, fumador moderado o fumador empedernido. Si se determina que el subconjunto de los fumadores sea un evento, entonces la totalidad de los no fumadores corresponderá a un evento diferente, también subconjunto de $S$, que se denomina complemento del conjunto de fumadores.

\begin{definicion}{}{}
    El complemento de un evento $A$ respecto de $S$ es el subconjunto de todos los elementos de $S$ que no están en $A$. Denotamos el complemento de $A$ mediante el símbolo $A^C$.
\end{definicion}

\newpage

\begin{examplebox}{}{}
    Sea $R$ el evento de que se seleccione una carta roja de una baraja ordinaria de 52 cartas, y sea $S$ toda la baraja. Entonces $R^C$ es el evento de que la carta seleccionada de la baraja no sea una roja sino una negra.
\end{examplebox}

\begin{definicion}{}{}
    La intersección de dos eventos $A$ y $B$, que se denota con el símbolo $A \cap B$, es el evento que contiene todos los elementos que son comunes entre $A$ y $B$.
\end{definicion}

\begin{examplebox}{}{}
    Sea $E$ el evento de que una persona seleccionada al azar en un salón de clases sea estudiante de ingeniería, y sea $F$ el evento de que la persona sea mujer. Entonces $E \cap F$ es el evento de todas las estudiantes mujeres de ingeniería en el salón de clases.
\end{examplebox}

\begin{examplebox}{}{}
    Sean $V = \{ a, e, i, o, u \}$ y $C = \{l, r, s, t \}$; entonces, se deduce que $V \cap C = \varnothing$. Es decir, $V$ y $C$ no tienen elementos comunes, por lo tanto, no pueden ocurrir de forma simultánea.
\end{examplebox}

Para ciertos experimentos estadísticos no es nada extraño definir dos eventos, $A$ y $B$, que no pueden ocurrir de forma simultánea. Se dice entonces que los eventos $A$ y $B$ son mutuamente excluyentes. Expresado de manera más formal, tenemos la siguiente definición:

\begin{definicion}{}{}
    Dos eventos $A$ y $B$ son mutuamente excluyentes o disjuntos si $A \cap B = \varnothing$; es decir, si $A$ y $B$ no tienen elementos en común.
\end{definicion}

A menudo nos interesamos en la ocurrencia de al menos uno de dos eventos asociados con un experimento. Por consiguiente, en el experimento del lanzamiento de un dado, si
$$A = \{ 2, 4, 6 \} \quad \text{ y } \quad B = \{ 4, 5, 6 \},$$
podríamos estar interesados en que ocurran $A$ o $B$, o en que ocurran tanto $A$ como $B$. Tal evento, que se llama unión de $A$ y $B$, ocurrirá si el resultado es un elemento del subconjunto $\{2, 4, 5, 6\}$.

\begin{definicion}{}{}
    La unión de dos eventos $A$ y $B$, que se denota con el símbolo $A \cup B$, es el evento que contiene todos los elementos que pertenecen a $A$ o a $B$, o a ambos.
\end{definicion}

\begin{examplebox}{}{}
    Sea $P$ el evento de que un empleado de una empresa petrolera seleccionado al azar fume cigarrillos. Sea $Q$ el evento de que el empleado seleccionado ingiera bebidas alcohólicas. Entonces, el evento $P \cup Q$ es el conjunto de todos los empleados que beben o fuman, o que hacen ambas cosas.
\end{examplebox}

\section{Probabilidad de un evento}

Quizá fue la insaciable sed del ser humano por el juego lo que condujo al desarrollo temprano de la teoría de la probabilidad. En un esfuerzo por aumentar sus triunfos, algunos pidieron a los matemáticos que les proporcionaran las estrategias óptimas para los diversos juegos de azar. Algunos de los matemáticos que brindaron tales estrategias fueron Pascal, Leibniz, Fermat y James Bernoulli. Como resultado de este desarrollo inicial de la teoría de la probabilidad, la inferencia estadística, con todas sus predicciones y generalizaciones, ha rebasado el ámbito de los juegos de azar para abarcar muchos otros campos asociados con los eventos aleatorios, como la política, los negocios, el pronóstico del clima y la investigación científica. Para que estas predicciones y generalizaciones sean razonablemente precisas, resulta esencial la comprensión de la teoría básica de la probabilidad.

En el resto de este capítulo consideraremos sólo aquellos experimentos para los cuales el espacio muestral contiene un número finito de elementos. La probabilidad de la ocurrencia de un evento que resulta de tal experimento estadístico se evalúa utilizando un conjunto de números reales denominados \emph{probabilidades}, que van de 0 a 1. Para todo elemento en el espacio muestral asignamos una probabilidad tal que la suma de todas las probabilidades es 1. Si tenemos razón para creer que al llevar a cabo el experimento es bastante probable que ocurra cierto elemento, le tendríamos que asignar a éste una probabilidad cercana a 1. Por el contrario, si creemos que no hay probabilidades de que ocurra cierto elemento, le tendríamos que asignar a éste una probabilidad cercana a cero. En muchos experimentos, como lanzar una moneda o un dado, todos los elementos tienen la misma oportunidad de ocurrencia, por lo tanto, se les asignan probabilidades iguales. A los elementos fuera del espacio muestral, es decir, a los eventos simples que no tienen posibilidades de ocurrir, les asignamos una probabilidad de cero.

Para encontrar la probabilidad de un evento $A$ sumamos todas las probabilidades que se asignan a los elementos en $A$. Esta suma se denomina probabilidad de $A$ y se denota con $P(A)$.

Para medir la probabilidad de que ocurra un evento específico, necesitamos una herramienta matemática que nos permita cuantificar esta incertidumbre. Aquí es donde entra en juego la \emph{función de probabilidad}. Esta función nos proporciona una manera sistemática de asignar un valor numérico a la probabilidad de cada posible resultado de un experimento aleatorio. En términos simples, la función de probabilidad nos ayuda a medir cuán probable es que ocurra un evento dado.

Un modelo probabilístico para un experimento con un espacio muestral discreto se puede construir al asignar una probabilidad numérica a cada evento simple del espacio muestral $S$. Seleccionaremos este número, una medida de nuestra creencia en que ocurrirá el evento en una sola repetición del experimento, de forma tal que será consistente con el concepto de frecuencia relativa de probabilidad. Aun cuando la frecuencia relativa no da una definición rigurosa de probabilidad, cualquier definición aplicable al mundo real debe concordar con nuestra noción intuitiva de las frecuencias relativas de eventos. Al analizar el concepto de frecuencia relativa de probabilidad, vemos que deben cumplirse tres condiciones.
\begin{enumerate}
    \item La frecuencia relativa de que ocurra cualquier evento debe ser mayor o igual a cero. Una frecuencia relativa negativa no tiene sentido.
    \item La frecuencia relativa de todo el espacio muestral $S$ debe ser la unidad. Debido a que todo posible resultado del experimento es un punto en $S$, se deduce que $S$ debe ocurrir cada vez que se realice el experimento.
    \item Si dos eventos son mutuamente excluyentes, la frecuencia relativa de la unión de ambos es la suma de sus respectivas frecuencias relativas.
\end{enumerate}
Estas tres condiciones forman la base de la siguiente definición.

\begin{definicion}{}{def_func_prob}
    Sea $S$ un espacio muestral y $\wp(S)$ el conjunto potencia de $S$. Definimos la función de probabilidad $P: \wp(S) \longrightarrow [0, 1]$, tal que cumple los siguientes tres axiomas:
    \begin{enumerate}[label=\roman*., topsep=6pt, itemsep=0pt]
        \item Para cualquier evento $A$, se tiene que $P(A) \geq 0$.
        \item La suma de las probabilidades de todos los eventos posibles debe ser igual a 1, es decir, $P(S) = 1$.
        \item Si $A_1, A_2, A_3, \dots$ son eventos disjuntos, $\displaystyle P\left( \bigcup_{i = 1}^{\infty} A_i \right) = \sum_{i = 1}^{\infty} P(A_i)$.
    \end{enumerate}
\end{definicion}

Observemos que la probabilidad del conjunto vacío $\varnothing$ es igual a 0, es decir, $P(\varnothing) = 0$. Esto se debe a que el conjunto vacío no contiene ningún evento posible, por lo tanto, la probabilidad de que ocurra un evento en el conjunto vacío es nula. Si el lector aún no entiende lo anteriormente dicho, esto será mucho más claro con los ejemplos siguientes y cuando presentemos la definición \ref{definicion:probabilidadnN}.

\newpage

Podemos fácilmente demostrar que el axioma 3 de la definición \ref{definicion:def_func_prob}, que está expresado en términos de una sucesión infinita de eventos, implica una propiedad similar para una sucesión finita. Específicamente, si $A_1, A_2, \dots, A_n$ son eventos mutuamente excluyentes por pares, entonces
$$P\left( \bigcup_{i = 1}^{n} A_i \right) = \sum_{i = 1}^{n} P(A_i).$$
La definición expresa solo las condiciones que una asignación de probabilidades debe satisfacer: no indica cómo asignar probabilidades específicas a los eventos.

\begin{examplebox}{}{}
    Una moneda se lanza dos veces. ¿Cuál es la probabilidad de que salga al menos un sol (S)?
    
    \tcblower
    \solucion El espacio muestral para este experimento es
    $$S = \{ \text{SS, SA, AS, AA} \}.$$
    Si la moneda está balanceada, cada uno de estos resultados tendrá las mismas probabilidades de ocurrir. Por lo tanto, asignamos una probabilidad de $s$ a cada uno de los elementos. Entonces, $4s = 1$ o $s = 1/4$. Si $A$ representa el evento de que salga al menos un sol (S), entonces
    $$A = \{ \text{SS, SA, AS} \} \quad \text{ y } \quad P(A) = \frac{1}{4} + \frac{1}{4} + \frac{1}{4} = \frac{3}{4}.$$
\end{examplebox}

\begin{examplebox}{}{}
    Se carga un dado de forma que exista el doble de probabilidades de que salga un número par que uno impar. Si $E$ es el evento de que ocurra un número menor que 4 en un solo lanzamiento del dado, calcule $P(E)$.
    
    \tcblower
    \solucion Es claro que el espacio muestral es $S = \{1, 2, 3, 4, 5, 6\}$. Asignamos una probabilidad de $s$ a cada número impar y una probabilidad de $2s$ a cada número par. Como la suma de las probabilidades debe ser 1, tenemos
    $$s + 2s + s + 2s + s + 2s = 9s = 1,$$
    o bien, $s = 1/9$. Por lo tanto, asignamos probabilidades de $1/9$ y $2/9$ a cada número impar y par, respectivamente. Por consiguiente,
    $$E = \{ 1, 2, 3 \} \quad \text{ y } \quad P(E) = \frac{1}{9} + \frac{2}{9} + \frac{1}{9} = \frac{4}{9}.$$
\end{examplebox}

\begin{examplebox}{}{}
    En el ejemplo anterior, sea $A$ el evento de que resulte un número par y sea $B$ el evento de que resulte un número divisible entre 3. Calcule $P(A \cup B)$ y $P(A \cap B)$.
    
    \tcblower
    \solucion Es claro que los eventos son $A = \{2, 4, 6\}$ y $B = \{3, 6\}$. Así que tenemos,
    $$A \cup B = \{ 2, 3, 4, 6 \} \quad \text{ y } \quad A \cap B = \{ 6 \}.$$
    Al asignar una probabilidad de $1/9$ a cada número impar y de $2/9$ a cada número par, tenemos
    $$P(A \cup B) = \frac{2}{9} + \frac{1}{9} + \frac{2}{9} + \frac{2}{9} = \frac{7}{9} \quad \text{ y } \quad P(A \cap B) = \frac{2}{9}.$$
\end{examplebox}

Si el espacio muestral para un experimento contiene $N$ elementos, todos los cuales tienen las mismas probabilidades de ocurrir, asignamos una probabilidad igual a $1/N$ a cada uno de los $N$ elementos. La probabilidad de que cualquier evento $A$ contenga $n$ de estos $N$ elementos es entonces el cociente del número de elementos en $A$ y el número de elementos en $S$.

\begin{definicion}{}{probabilidadnN}
    Si un experimento puede dar como resultado cualquiera de $N$ diferentes resultados que tienen las mismas probabilidades de ocurrir, y si exactamente $n$ de estos resultados corresponden al evento $A$, entonces la probabilidad del evento $A$ es
    $$P(A) = \frac{n}{N}.$$
\end{definicion}

\begin{examplebox}{}{}
    A una clase de estadística para ingenieros asisten 25 estudiantes de ingeniería industrial, 10 de ingeniería mecánica, 10 de ingeniería eléctrica y 8 de ingeniería civil. Si el profesor elige al azar a un estudiante para que conteste una pregunta, ¿qué probabilidades hay de que el elegido sea
    \begin{enumerate}[label=\alph*), topsep=6pt, itemsep=0pt]
        \item estudiante de ingeniería industrial,
        \item estudiante de ingeniería civil o estudiante de ingeniería eléctrica?
    \end{enumerate}
    \tcblower
    \solucion Asignemos las especialidades de los estudiantes de ingeniería industrial, mecánica, eléctrica y civil por las letras $I$, $M$, $E$ y $C$, respectivamente. El grupo está integrado por 53 estudiantes y todos tienen las mismas probabilidades de ser seleccionados.
    \begin{enumerate}[label=\alph*), topsep=6pt, itemsep=0pt]
        \item Como 25 de los 53 individuos estudian ingeniería industrial, la probabilidad del evento $I$, es decir, la de elegir al azar a alguien que estudia ingeniería industrial, es
        $$P(I) = \frac{25}{53}.$$
        \item Como 18 de los 53 estudiantes son de las especialidades de ingeniería civil o eléctrica, se deduce que
        $$P(C \cup E) = \frac{18}{53}.$$
    \end{enumerate}
\end{examplebox}